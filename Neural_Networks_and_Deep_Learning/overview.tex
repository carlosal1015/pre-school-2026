\begin{abstract}

El aprendizaje automático es un poderoso conjunto de técnicas que
permiten que las computadoras aprendan de los datos en lugar de
que un programa humano experto tenga un comportamiento a mano.
Las redes neuronales son una clase de algoritmo de aprendizaje
automático originalmente inspirado en el cerebro, pero que
recientemente han tenido mucho éxito en aplicaciones prácticas.
Están en el corazón de los sistemas de producción en compañías
como Google y Facebook para el reconocimiento de rostros, el
habla a texto y la comprensión del idioma.

\

Este curso ofrece una visión general de las ideas fundamentales
y los avances recientes en algoritmos de redes neuronales.
Aproximadamente los primeros $2/3$ del curso se centran en el
aprendizaje supervisado: capacitación de la red para producir un
comportamiento específico cuando uno tiene muchos ejemplos
etiquetados de ese comportamiento.
El último $1/3$ se enfoca en el aprendizaje no supervisado y
el aprendizaje de refuerzo.

\

\begin{figure}[H]
	\centering
	\includegraphics[width=0.5\paperwidth]{geoffrey-hinton}
	\caption*{Profesor Geoffrey Hinton de la Universidad de Toronto.
		Premio Alan Turing 2018 por avances conceptuales y de ingeniería
		que han hecho de las redes neuronales profundas un componente
		crítico de la informática.}
\end{figure}

\end{abstract}

\section*{Prerrequisitos}

Este es un segundo curso de aprendizaje automático, por lo que tiene
algunos requisitos previos sustanciales.
Se cumplirán estos requisitos previos, incluso para los estudiantes
de posgrado.

\begin{description}
	\item[Cálculo avanzado] Gradiente.
	\item[Álgebra lineal] Matrices.
   \item[Aprendizaje automático] Regresión lineal, regresión
   logística, principio de máxima verosimilitud, análisis de
   componentes principales, algoritmo esperanza-maximización.
\end{description}

\section*{Lecturas}


No hay un libro de texto requerido para la clase.
Se pueden asignar algunas lecturas pequeñas si surge la necesidad.
Estas lecturas obligatorias estarán disponibles en la web de forma
gratuita.
También hay algunos recursos relevantes que están disponibles
gratuitamente en línea.
Trataremos de proporcionar enlaces conferencia por conferencia.

\section*{Recursos de computación}

\begin{figure}[H]
	\vfill
	\begin{subfigure}{.5\textwidth}
		\centering
		\includesvg[width=.4\linewidth]{colab}
		\caption*{Google Colaboratory.}
		\label{fig:colab}
	\end{subfigure}
	\begin{subfigure}{.5\textwidth}
		\centering
		\includesvg[width=.4\linewidth]{compute-engine}
		\caption*{Google Compute Engine.}
		\label{fig:gce}
	\end{subfigure}
	\caption*{Dos herramientas de Google Inc.
		Izquierda: \url{https://colab.research.google.com}.
		Derecha: \url{https://cloud.google.com/compute}.}
	\label{fig:google}
\end{figure}

\section*{Sílabus}

Por favor vea el sílabo \href{https://csc413-2020.github.io/assets/misc/syllabus.pdf}{aquí}.
