\section{Motivación}

\subsection{¿Por qué el aprendizaje automático?}


Piense en algunas de las cosas que hacemos sin esfuerzo en el día a
día: reconocer visualmente a las personas, los lugares y las cosas,
recoger objetos, comprender el lenguaje hablado, etc.
¿Cómo programarías una máquina para hacer estas cosas?
Desafortunadamente, es difícil dar un programa paso a paso, ya que
tenemos muy poca conciencia introspectiva del funcionamiento de
nuestras mentes.
¿Cómo reconoces a tu mejor amigo?
¿Exactamente qué rasgos faciales eliges?
Los investigadores de Inteligencia Artificial intentaron durante
décadas idear procedimientos computacionales para este tipo de
tareas, y resultó frustrantemente difícil.

El aprendizaje automático tiene un enfoque diferente: recopilar
muchos datos y hacer que un algoritmo descubra automáticamente un
buen comportamiento a partir de los datos.
Si está intentando escribir un programa para distinguir diferentes
categorías de objetos (árbol, perro, etc.), primero puede recopilar
un conjunto de datos de imágenes de cada tipo de objeto y luego usar
un algoritmo de aprendizaje automático para entrenar un modelo
(como una red neuronal) para clasificar una imagen como una categoría
u otra.
Tal vez aprenderá a ver de una manera análoga al sistema visual
humano, o tal vez tenga un enfoque completamente diferente.
De cualquier manera, todo el proceso puede ser mucho más fácil que
especificar todo a mano.

Además de ser más fácil, hay muchas otras razones por las que
podríamos querer usar el aprendizaje automático para resolver un
problema dado:

\begin{itemize}
    \item .
\end{itemize}

\subsection{¿En qué se diferencia el Aprendizaje automático de la estadística?}


Muchos de los algoritmos empleados provienen de la estadística:
regresión lineal, análisis de componentes principales (PCA),
estimación de máxima probabilidad, estimación de parámetros bayesianos
y maximización de expectativas.
La mayoría de las técnicas usadas en el aprendizaje automático desde
las más básicas hasta las más avanzadas son estadísticos es por ello
que ambos campos tiene mucha similitud ya que ambos buscan aprender
de los datos.
Las opiniones sobre cuál es la diferencia difieren, pero en términos
generales,la estadística está motivado por guiar la toma de decisiones
humanas, mientras que el aprendizaje automático está motivado por
agentes autónomos.

\subsection{¿Por qué un curso sobre redes neuronales?}


Las redes neuronales son un enfoque particular del aprendizaje
automático, inspiradas en como procesa el cerebro la información.
Una red neuronal está compuesto por una gran cantidad de unidades,
cada una de las cuales realiza cálculos muy simples, pero que producen
comportamientos sofisticados en conjunto.
Este curso se centra en redes neuronales por varias razones:

\begin{itemize}
   \item La redes neuronales son la base de los sistemas de
   reconocimiento de voz, traducción, clasificación de resultados
   de búsqueda, reconocimiento facial,
   análisis de sentimientos, búsqueda de imágenes, y muchas otras
   aplicaciones.

   \item Existen potentes paquetes de software como \texttt{Caffe},
   \texttt{Theano}, \texttt{Torch} y \texttt{TensorFlow}, que nos
   permite implementar rápidamente algoritmos de aprendizaje
   sofisticados.

   \item Muchos de los algoritmos importantes son mucho más simples
   de explicar, en comparación con otros subcampos del aprendizaje
   automático.
\end{itemize}

Esta clase es muy inusual entre las clases de pregrado, ya que cubre
técnicas modernas de investigación, es decir, algoritmos introducidos
en los últimos 5 años.
Es sorprendente que con menos de una página de código, podamos
construir algoritmos de aprendizaje más potentes que los mejores
investigadores habían ideado a partir de hace 5 años.

\section{Tipos de Aprendizaje automático}


En términos generales, hay tres diferentes tipos de aprendizaje
automático:

\begin{description}
	\item[Aprendizaje supervisado] Tenemos ejemplos del comportamiento
   deseado. Por ejemplo, si estamos tratando de entrenar una red
   neuronal para distinguir los autos y camiones, recolectamos
   imágenes de autos y camiones, y rotulamos cada uno como un
   automóvil o un camión.

\begin{multicols}{2}
	\begin{itemize}
		\item \href{https://es.wikipedia.org/wiki/K_vecinos_m%C3%A1s_pr%C3%B3ximos}{$k$ vecinos más próximos}.
		\item \href{https://es.wikipedia.org/wiki/Regresi%C3%B3n_lineal}{regresión lineal}.
		\item \href{https://es.wikipedia.org/wiki/Regresi%C3%B3n_log%C3%ADstica}{regresión logística}.
		\item \href{https://es.wikipedia.org/wiki/M%C3%A1quinas_de_vectores_de_soporte}{máquinas de soporte vectorial}.
		\item \href{https://es.wikipedia.org/wiki/Aprendizaje_basado_en_%C3%A1rboles_de_decisi%C3%B3n}{aprendizaje basado en árboles de decisión}.
		\item \href{https://es.wikipedia.org/wiki/Random_forest}{bosques aleatorios}.
		\item \href{https://es.wikipedia.org/wiki/Red_neuronal_artificial}{red neuronal artificial}.
	\end{itemize}
\end{multicols}

\begin{figure}[H]
	\vfill
	\begin{subfigure}{.5\textwidth}
		\centering
		\includegraphics[width=\linewidth]{spam-classification}
		\caption*{}
		\label{fig:spam}
	\end{subfigure}
	\begin{subfigure}{.5\textwidth}
		\centering
		\includesvg[width=\linewidth]{regression}
		\caption*{}
		\label{fig:regression}
	\end{subfigure}
   \caption{Izquierda: Un conjunto etiquetados de entrenamiento para
   el aprendizaje supervisado, por ejemplo la clasificación de spam.
	Derecha: Regresión.}
	\label{fig:supervised-learning}
\end{figure}

   \item[Aprendizaje reforzado] No tenemos ejemplos del
   comportamiento, pero tenemos algún método para determinar qué tan
   bueno es un comportamiento esto se conoce como una
   \emph{señal de recompensa}.
	(Por analogía, es como entrenar perros para realizar trucos). Un
	ejemplo sería entrenar un agente para jugar videojuegos, donde la
	señal de recompensa es la del puntuación del jugador.

	\begin{figure}[H]
		\centering
		\includegraphics[width=0.5\paperwidth]{reinforcement-learning}
		\caption{Aprendizaje reforzado.}
		\label{fig:reinforcement-learning}
	\end{figure}

	\item[Aprendizaje no supervisado] No tenemos etiquetas ni señal de
   recompensa. Solo tenemos un montón de datos y queremos buscar
   patrones en los datos. Por ejemplo, quizás tengamos muchos ejemplos
   de pacientes con autismo, y desea identificar diferentes subtipos
   de la condición.

	\begin{figure}[H]
		\centering
		\includegraphics[width=0.5\paperwidth]{train-set}
		\caption{Un conjunto no etiquetados de entrenamiento para el aprendizaje no supervisado.}
		\label{fig:train-set}
	\end{figure}

	\begin{itemize}
		\item \href{https://es.wikipedia.org/wiki/An%C3%A1lisis_de_grupos}{Agrupamiento}.
			\begin{itemize}
				\item \href{https://es.wikipedia.org/wiki/K-medias}{$K$-medias}.
				\item \href{https://es.wikipedia.org/wiki/Agrupamiento_jer%C3%A1rquico}{Agrupamiento jerárquico}.
				\item \href{https://es.wikipedia.org/wiki/Algoritmo_esperanza-maximizaci%C3%B3n}{Algoritmo esperanza-maximización}.
			\end{itemize}
		\item Visualización y reducción de dimensionalidad
			\begin{itemize}
				\item \href{https://es.wikipedia.org/wiki/An%C3%A1lisis_de_componentes_principales}{Análisis de componentes principales}.
				\item \href{https://en.wikipedia.org/wiki/Kernel_principal_component_analysis}{Kernel principal component analysis}.
				\item \href{https://en.wikipedia.org/wiki/Nonlinear_dimensionality_reduction#Locally-linear_embedding}{Locally-linear embedding}
				\item \href{https://en.wikipedia.org/wiki/T-distributed_stochastic_neighbor_embedding}{$t$-distributed stochastic neighbor embedding}
			\end{itemize}
		\item \href{https://es.wikipedia.org/wiki/Reglas_de_asociaci%C3%B3n}{Reglas de asociación}
			\begin{itemize}
				\item \href{https://es.wikipedia.org/wiki/Algoritmo_apriori}{Algoritmo apriori}
				\item \href{https://en.wikipedia.org/wiki/Association_rule_learning#Eclat_algorithm}{Eclat algorithm}
			\end{itemize}
	\end{itemize}

	\begin{figure}[H]
		\centering
		\includegraphics[width=0.5\paperwidth]{clustering}
		\caption{Agrupación.}
		\label{fig:clustering}
	\end{figure}
\end{description}
